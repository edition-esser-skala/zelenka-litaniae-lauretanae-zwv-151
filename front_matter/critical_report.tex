\documentclass{ees}

\shorttitle{Lit. lauretanæ}

\begin{document}

\eesTitlePage

\eesCriticalReport{
  1 & 27   & S    & 8th \sixteenthNote\ in \A1: \sharp d″16 \\
    & 34   & org  & 1st half of bar illegible in \A1,
                    here modeled on bar 71 \\
    & 45f  & ob 2, vl 2 & in \A1 unison with T, here adapted to vl 1 \\
    & 45f  & vla  & in \A1 unison with T, here adapted to vl 1 \\
    & 91   & vla  & last \eighthNote\ missing in \A1 \\
    & 124  & S, vl 2 & 3rd \eighthNote\ in \A1: \sharp f′16–e′16 \\
  \midrule
  2 & –    & ob   & In \A1, the directives “Tutti” and “Vv.” in the violin parts
                    indicate the beginning and end of segments where the oboes
                    should play unison with the violins. Based on these
                    directives, the oboe parts of this edition have been
                    assembled. Nevertheless, the directives are retained
                    in the violin parts. If a chord appears in the violin part,
                    only the highest note is retained for the oboe part. \\
    & –    & ob 1 & The following bars have been emended to accommodate
                    the oboe’s range: 18, 58, and 88. \\
    & –    & ob 2 & The following bars have been emended to accommodate
                    the oboe’s range: 18, 58, and 88. \\
  \midrule
  3 & 26   & ob   & in \A1 unison with vl, here modeled on bars 41/81 \\
    & 37   & ob 1 & in \A1 unison with S, here unison with vl 1 \\
    & 37   & ob 2 & in \A1 unison with S, here unison with vl 2 \\
    & 46   & vl 1 & 2nd \eighthNote\ in \A1: g′16–g′16 \\
    & 66   & ob   & in \A1 unison with vl, here modeled on bars 41/81 \\
    & 71f  & ob   & in \A1 unison with vl, here modeled on bars 41/81 \\
    & 73   & vl, vla & in \A1 equal to bar 146, adapted to org \\
    & 81   & ob 2 & in \A1 unison with vl 2, here modeled on bar 41 \\
  \midrule
  4 & 7    & vl 2 & 1st, 5th and 9th \sixteenthNote\ in \A1 unison with vla \\
    & 10  & vl 2 & 9th \sixteenthNote\ in \A1 unison with vla \\
    & 12  & vl 2 & 9th \sixteenthNote\ in \A1 unison with vla \\
    & 13  & vl 2 & 1st and 9th \sixteenthNote\ in \A1 unison with vla \\
    & 102  & org  & 3rd \quarterNote\ in \A1: f4 \\
    & 126  & vl 2 & adapted to bars 1–7 \\
  \midrule
  5 & –    & –    & \A1\ only contains chorus and org. Here, the bass figures
                    of the \textit{Kyrie} are used. In the instruments,
                    the rhythm of the subject has been adapted accordingly.
                    Minor differences to the \textit{Kyrie} in S (bar 44),
                    A (bars 10, 12, 26, and 29),
                    T (bars 25, 31, 33, and 34),
                    B (bars 9 and 54–56), and
                    org (bars 5, 9, and 54–56) were also incorporated
                    into the instrumental parts. \\
    & 1 & A     & 2nd \halfNote\ in \A1: \sharp f′4–\sharp f′8–d′8 \\
}

\eesToc{}

\eesScore

\end{document}
